\newglossaryentry{template unit}{
  name=template unit,
  description={
      A systemd unit file that defines a generic service or target using a
      parameterized name. Template unit files end in \texttt{@.service} (or other
      unit type suffixes) and are instantiated by supplying an instance name,
      e.g., \texttt{myservice@foo.service}. Inside the template, the instance name
      can be referenced using specifiers such as \texttt{\%i}. This mechanism
      supports scalable patterns such as one unit per user, device, or interface
    }
}

\newglossaryentry{unit-section}{
  name=unit section,
  description={
      A logical partition within a systemd unit file. Unit files follow an
      INI-style format, where sections are enclosed in square brackets such
      as \texttt{[Unit]}, \texttt{[Service]}, or \texttt{[Install]}. Each
      section groups together related configuration directives. Some
      sections are common to all units (e.g., \texttt{[Unit]},
      \texttt{[Install]}), while others are specific to the unit type (e.g.,
      \texttt{[Service]} for \texttt{.service} units, \texttt{[Timer]} for
      \texttt{.timer} units). Systemd parses these sections to build the
      dependency graph and execution behavior of the unit
    }
}

\newglossaryentry{target-unit}{
  name=target unit,
  description={
      A special kind of systemd unit (with suffix \texttt{.target}) that
      groups other units together. Targets do not start processes themselves;
      instead, they serve as synchronization points and activation milestones
      in the boot or user session. Examples include \texttt{multi-user.target},
      \texttt{graphical.target}, and \texttt{default.target}. They are often
      used as \texttt{WantedBy=} or \texttt{RequiredBy=} anchors in other unit
      files.
    }
}

\newglossaryentry{activation-queue}{
  name=activation queue,
  description={
      The ordered set of jobs that systemd creates when resolving a unit
      start, stop, or isolate request. The activation queue is built from the
      dependency graph, with ordering constraints (\texttt{Before=},
      \texttt{After=}) applied. It represents the sequence of unit state
      changes that systemd will dispatch to reach the requested target state
    }
}
