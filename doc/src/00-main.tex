\documentclass[openany, 12pt]{book}
\makeindex

\usepackage{amsmath}
\usepackage{amssymb}
\usepackage{booktabs}
\usepackage[dvipsnames]{xcolor}
\usepackage{enumitem}
\usepackage[toc]{glossaries}
\usepackage{graphicx}
\usepackage[citecolor=blue,colorlinks=true, linkcolor=blue, urlcolor=blue]{hyperref}
\usepackage{makeidx}
\usepackage[margin=0.8in]{geometry}
\usepackage{mathrsfs}
\usepackage{minted}
\usepackage{multicol}
\usepackage[style=authortitle]{biblatex}
\usepackage[T1]{fontenc}
\usepackage{tcolorbox}
\usepackage{todonotes}
\usepackage{tikz}
\usepackage{titlesec}
\usepackage[table]{xcolor}

\usetikzlibrary{arrows}
\usetikzlibrary{arrows.meta}
\usetikzlibrary{automata}
\usetikzlibrary{calc}
\usetikzlibrary{fit}
\usetikzlibrary{petri}
\usetikzlibrary{positioning}

\tcbuselibrary{breakable}
\tcbuselibrary{listings}
\tcbuselibrary{minted}
\tcbuselibrary{skins}
\tcbuselibrary{theorems}

\newcounter{filePrg}

% \addbibresource{biblio.bib}
\setlength{\parindent}{0pt}

\renewcommand{\emph}[1]{\textit{#1}}
\setlength{\parindent}{0pt}

\renewcommand*{\glstextformat}[1]{\textbf{#1}}
\newcommand\setboxcounter[2]{\setcounter{tcb@cnt@#1}{#2}}
\setlength{\parindent}{10pt}
\newcommand{\set}[1]{\{#1\}}

\definecolor{CaribbeanBlue}{RGB}{0, 206, 209} % Define Caribbean Blue
\NewTcbTheorem[list inside=definition]{definition}
{Definition}{
	breakable,
	colback=CaribbeanBlue!05,
	colframe=CaribbeanBlue!35!black,
	fonttitle=\bfseries}{th}

\NewTcbTheorem[list inside=intuition]{intuition}{Intuition}{
	breakable,
	colback=blue!5,
	colframe=blue!35!black,
	fonttitle=\bfseries}{th}

\NewTcbTheorem{example}{Example}{
	breakable,
	colback=white,
	colframe=green!35!black,
	fonttitle=\bfseries}{th}

\NewTcbTheorem{verify}{Verify}{
	breakable,
	float,
	colback=red!5,
	colframe=red!35!black,
	fonttitle=\bfseries}{th}

\NewTcbTheorem[list inside=theorem]{theorem}{Theorem}{
	breakable,
	colback=gray!10,
	colframe=gray!35!black,
	fonttitle=\bfseries}{th}

\NewTcbTheorem[
	list inside=exercise,
	number within=section
]
{exercise}{Exercise}{
	breakable,
	colback=white,
	colframe=black,
	fonttitle=\bfseries}{th}

\newcommand{\hask}[1]{\mintinline{haskell}{#1}}

\newenvironment{alist}
{\begin{enumerate}[label={*}, leftmargin=*, itemsep=0pt, parsep=0pt]}
		{\end{enumerate}}

\newenvironment{blist}
{\begin{enumerate}[label={}, leftmargin=*, itemsep=0pt, parsep=0pt]}
		{\end{enumerate}}


\renewcommand{\thesection}{\arabic{section}}
\tcbset{enhanced jigsaw}

\newtcbinputlisting{\codeFromFile}[2]{
	listing file={#1},
	listing engine=minted,
	minted style=colorful,
	minted language=haskell,
	minted options={breaklines,linenos,numbersep=3mm},
	colback=blue!5!white,colframe=blue!75!black,listing only,
	left=5mm,enhanced,
	title={#2},
	overlay={\begin{tcbclipinterior}\fill[red!20!blue!20!white] (frame.south west)
				rectangle ([xshift=5mm]frame.north west);\end{tcbclipinterior}}
}

\newtcblisting{haskell}[1]
{
	listing engine=minted,
	minted style=colorful,
	minted language=haskell,
	minted options={breaklines,linenos,numbersep=3mm},
	colback=blue!5!white,colframe=blue!75!black,listing only,
	left=5mm,enhanced,
	title={#1},
	overlay={\begin{tcbclipinterior}\fill[red!20!blue!20!white] (frame.south west)
				rectangle ([xshift=5mm]frame.north west);\end{tcbclipinterior}}
}

\newtcblisting{shell}[1]
{
	listing engine=minted,
	minted style=colorful,
	minted language=shell,
	minted options={breaklines,linenos,numbersep=3mm},
	colback=blue!5!white,colframe=blue!75!black,listing only,
	left=5mm,enhanced,
	title={#1},
	overlay={\begin{tcbclipinterior}\fill[red!20!blue!20!white] (frame.south west)
				rectangle ([xshift=5mm]frame.north west);\end{tcbclipinterior}}
}

\title{SystemD}
\author{Idris}
\date{June 2025}

%chktex-file 1

\begin{document}
\tableofcontents

\chapter{Background}

Let
\begin{align*}
	U = \set{u_1, u_2, \dots, u_n}
\end{align*}

be the finite set of all systemd units in the system. Each element $u_i \in U$
represents a unique systemd unit, such as a service, target, socket, or timer.

We define dependency and ordering relationships as binary relations on $U
	\times U$, which we will formalize in the following sections.

\begin{definition}{unit states}{}
	Let $S$ be the set of possible states of a unit:
	\begin{align*}
		S = \set{ \mathsf{inactive},\ \mathsf{activating},\ \mathsf{active},\ \mathsf{deactivating},\ \mathsf{failed}}
	\end{align*}
	Define a state function
	\begin{align*}
		a : U \to S
	\end{align*}
	which maps each unit to its current state.
\end{definition}

\begin{definition}{started predicate}{}
	Let
	\begin{align*}
		\mathsf{started}(u) \iff a(u) \in \set{ \mathsf{activating},\ \mathsf{active},\ \mathsf{deactivating}}
	\end{align*}
	This captures units that are in the process of starting, running, or
	stopping — i.e., **not inactive or failed**.
\end{definition}

\begin{definition}{wants predicate}{}
	Given a relation
	\begin{align*}
		\text{Wants} \subseteq U \times U
	\end{align*}
	we define the systemd behavior as:

	\begin{quote}
		If $\mathsf{started}(a)$ becomes true, then for each $(a, b) \in
			\text{Wants}$, systemd attempts $\mathsf{start}(b)$, regardless of
		whether $\mathsf{start}(b)$ eventually succeeds or fails.
	\end{quote}

	This defines `Wants` as a rule over the state predicate $\mathsf{started}$,
	rather than an abstract dependency graph.
\end{definition}

% \printbibliography{}
% \printindex{}
\end{document}
