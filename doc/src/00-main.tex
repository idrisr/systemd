\documentclass[openany, 12pt]{book}
\makeindex

\usepackage{amsmath}
\usepackage{amssymb}
\usepackage{booktabs}
\usepackage[dvipsnames]{xcolor}
\usepackage{enumitem}
\usepackage[toc]{glossaries}
\usepackage{graphicx}
\usepackage[citecolor=blue,colorlinks=true, linkcolor=blue, urlcolor=blue]{hyperref}
\usepackage{makeidx}
\usepackage[margin=0.8in]{geometry}
\usepackage{mathrsfs}
\usepackage{minted}
\usepackage{multicol}
\usepackage[style=authortitle]{biblatex}
\usepackage[T1]{fontenc}
\usepackage{tcolorbox}
\usepackage{todonotes}
\usepackage{tikz}
\usepackage{titlesec}
\usepackage[table]{xcolor}

\usetikzlibrary{arrows}
\usetikzlibrary{arrows.meta}
\usetikzlibrary{automata}
\usetikzlibrary{calc}
\usetikzlibrary{fit}
\usetikzlibrary{petri}
\usetikzlibrary{positioning}

\tcbuselibrary{breakable}
\tcbuselibrary{listings}
\tcbuselibrary{minted}
\tcbuselibrary{skins}
\tcbuselibrary{theorems}

\newcounter{filePrg}

% \addbibresource{biblio.bib}
\setlength{\parindent}{0pt}

\renewcommand{\emph}[1]{\textit{#1}}
\setlength{\parindent}{0pt}

\renewcommand*{\glstextformat}[1]{\textbf{#1}}
\newcommand\setboxcounter[2]{\setcounter{tcb@cnt@#1}{#2}}
\setlength{\parindent}{10pt}
\newcommand{\set}[1]{\{#1\}}

\definecolor{CaribbeanBlue}{RGB}{0, 206, 209} % Define Caribbean Blue
\NewTcbTheorem[list inside=definition]{definition}
{Definition}{
	breakable,
	colback=CaribbeanBlue!05,
	colframe=CaribbeanBlue!35!black,
	fonttitle=\bfseries}{th}

\NewTcbTheorem[list inside=intuition]{intuition}{Intuition}{
	breakable,
	colback=blue!5,
	colframe=blue!35!black,
	fonttitle=\bfseries}{th}

\NewTcbTheorem{example}{Example}{
	breakable,
	colback=white,
	colframe=green!35!black,
	fonttitle=\bfseries}{th}

\NewTcbTheorem{verify}{Verify}{
	breakable,
	float,
	colback=red!5,
	colframe=red!35!black,
	fonttitle=\bfseries}{th}

\NewTcbTheorem[list inside=theorem]{theorem}{Theorem}{
	breakable,
	colback=gray!10,
	colframe=gray!35!black,
	fonttitle=\bfseries}{th}

\NewTcbTheorem[
	list inside=exercise,
	number within=section
]
{exercise}{Exercise}{
	breakable,
	colback=white,
	colframe=black,
	fonttitle=\bfseries}{th}

\newcommand{\hask}[1]{\mintinline{haskell}{#1}}

\newenvironment{alist}
{\begin{enumerate}[label={*}, leftmargin=*, itemsep=0pt, parsep=0pt]}
		{\end{enumerate}}

\newenvironment{blist}
{\begin{enumerate}[label={}, leftmargin=*, itemsep=0pt, parsep=0pt]}
		{\end{enumerate}}


\renewcommand{\thesection}{\arabic{section}}
\tcbset{enhanced jigsaw}

\newtcbinputlisting{\codeFromFile}[2]{
	listing file={#1},
	listing engine=minted,
	minted style=colorful,
	minted language=haskell,
	minted options={breaklines,linenos,numbersep=3mm},
	colback=blue!5!white,colframe=blue!75!black,listing only,
	left=5mm,enhanced,
	title={#2},
	overlay={\begin{tcbclipinterior}\fill[red!20!blue!20!white] (frame.south west)
				rectangle ([xshift=5mm]frame.north west);\end{tcbclipinterior}}
}

\newtcblisting{haskell}[1]
{
	listing engine=minted,
	minted style=colorful,
	minted language=haskell,
	minted options={breaklines,linenos,numbersep=3mm},
	colback=blue!5!white,colframe=blue!75!black,listing only,
	left=5mm,enhanced,
	title={#1},
	overlay={\begin{tcbclipinterior}\fill[red!20!blue!20!white] (frame.south west)
				rectangle ([xshift=5mm]frame.north west);\end{tcbclipinterior}}
}

\newtcblisting{shell}[1]
{
	listing engine=minted,
	minted style=colorful,
	minted language=shell,
	minted options={breaklines,linenos,numbersep=3mm},
	colback=blue!5!white,colframe=blue!75!black,listing only,
	left=5mm,enhanced,
	title={#1},
	overlay={\begin{tcbclipinterior}\fill[red!20!blue!20!white] (frame.south west)
				rectangle ([xshift=5mm]frame.north west);\end{tcbclipinterior}}
}

\title{SystemD}
\author{Idris}
\date{June 2025}

%chktex-file 1

\begin{document}
\tableofcontents
\listoftodos

\chapter{Unit Skeleton}
\begin{alist}
	\item all units have a Unit section
	\item contain a description, reference to documentation and dependencies on
	other unites
\end{alist}

\chapter{Ordering}

Some of the relations form the dependency graph. Then for any node, we can
create a closure over the set of dependent nodes. Then from that closure we can
create a time ordering.

\section{Unit Relations}

\todo[inline]{Improve the order definitions using predicates over the states}
\begin{definition}{Topological Order}{}
	A topological order is a total ordering over a subset of units such that for
	any
	\begin{align*}
		(a, b) \in R \subseteq U \times U,
	\end{align*}
	if $a$ must causally precede $b$ (e.g., via \texttt{Wants} or \texttt{Requires}),
	then $a$ appears before $b$ in the ordering.

	This applies when $R$ forms a directed acyclic graph (DAG), and is used to
	compute valid execution sequences from dependency relations.
\end{definition}

\begin{definition}{Time Order}{}
	A time order is a scheduling constraint over unit startup.

	Given
	\begin{align*}
		(a, b) \in R \subseteq U \times U,
	\end{align*}
	the relation enforces that if both $a$ and $b$ are scheduled to start, then
	$\mathsf{start}(a)$ is delayed until $\mathsf{start}(b)$ has finished
	(successfully or not).

	Time order does not imply dependency; it only constrains relative ordering.
\end{definition}

\begin{definition}{Lifetime Binding}{}
	A lifetime binding is a coupling between the runtime lifecycles of units.

	Given
	\begin{align*}
		(a, b) \in R \subseteq U \times U,
	\end{align*}
	the relation enforces that events in the lifetime of $b$ (e.g., stopping,
	restarting) propagate to $a$.

	Examples include \texttt{BindsTo} (stop cascades) and \texttt{PartOf}
	(restart cascades).
\end{definition}

\begin{table}[h]
	\centering
	\begin{tabular}{lccc l}
		\toprule
		\textbf{Relation} & \textbf{Topo.} & \textbf{Time} & \textbf{Bind / Reverse} & \textbf{Note}            \\
		\midrule
		Wants             & \checkmark     &               &                         & Soft dependency          \\
		Requires          & \checkmark     &               &                         & Hard dependency (fatal)  \\
		Requisite         & \checkmark     &               &                         & Must succeed pre-start   \\
		Conflicts         & \checkmark     &               &                         & Mutual exclusion         \\
		After             &                & \checkmark    &                         & Starts after if present  \\
		Before            &                & \checkmark    &                         & Starts before if present \\
		BindsTo           &                &               & \checkmark              & Stop cascades            \\
		PartOf            &                &               & \checkmark              & Restart/stop cascades    \\
		JoinsNamespaceOf  &                &               & \checkmark              & Shares runtime namespace \\
		RequiredBy        &                &               & $\leftarrow$            & Reverse dependency       \\
		\bottomrule
	\end{tabular}
	\caption{Systemd relations classified by axis with compact notes}
\end{table}


Let $U$ be the finite set of all systemd units.

We define three fundamental binary relations on $U \times U$:

\begin{align*}
	\text{Wants}   & \subseteq U \times U &  & \text{(Dependency)}        \\
	\text{After}   & \subseteq U \times U &  & \text{(Time Ordering)}     \\
	\text{BindsTo} & \subseteq U \times U &  & \text{(Lifecycle Binding)}
\end{align*}

\begin{description}
	\item[Wants:] If $(a, b) \in \text{Wants}$, then starting $a$ causes systemd
	      to attempt to start $b$. No ordering is implied.

	\item[After:] If $(a, b) \in \text{After}$, and both $a$ and $b$ are being
	      started, then systemd delays starting $a$ until after $b$ has finished
	      starting (success or failure).

	\item[BindsTo:] If $(a, b) \in \text{BindsTo}$, then stopping $b$ causes
	      systemd to stop $a$. The lifecycle of $a$ is bound to that of $b$.
\end{description}


\chapter{Relations}
\begin{enumerate}[label = {(\arabic*)}]
	\item Wants
	\item Requires
	\item Before
	\item After
	\item Conflicts
\end{enumerate}

Let
\begin{align*}
	U = \set{u_1, u_2, \dots, u_n}
\end{align*}

be the finite set of all systemd units in the system. Each element $u_i \in U$
represents a unique systemd unit, such as a service, target, socket, or timer.

We define dependency and ordering relationships as binary relations on $U
	\times U$, which we will formalize in the following sections.

\begin{definition}{Activation Queue}{}

\end{definition}

\begin{definition}{unit states}{}
	Let $S$ be the set of possible states of a unit:
	\begin{align*}
		S = \set{ \mathsf{inactive},\ \mathsf{activating},\ \mathsf{active},\ \mathsf{deactivating},\ \mathsf{failed}}
	\end{align*}
	Define a state function
	\begin{align*}
		a : U \to S
	\end{align*}
	which maps each unit to its current state.
\end{definition}

\begin{definition}{started predicate}{}
	Let
	\begin{align*}
		\mathsf{started}(u) \iff a(u) \in \set{ \mathsf{activating},\ \mathsf{active},\ \mathsf{deactivating}}
	\end{align*}
	This captures units that are in the process of starting, running, or
	stopping — i.e., **not inactive or failed**.
\end{definition}

\section{Wants}
\begin{definition}{wants predicate}{}
	Given a relation
	\begin{align*}
		\text{Wants} \subseteq U \times U
	\end{align*}
	we define the systemd behavior as:

	\begin{quote}
		If $\mathsf{started}(a)$ becomes true, then for each $(a, b) \in
			\text{Wants}$, systemd attempts $\mathsf{start}(b)$, regardless of
		whether $\mathsf{start}(b)$ eventually succeeds or fails.
	\end{quote}

	This defines ``Wants'' as a rule over the state predicate $\mathsf{started}$,
	rather than an abstract dependency graph.

	A weaker form of Requires: unit is started if any in the list fail.
\end{definition}

\section{WantedBy}
\begin{definition}{wantedby predicate}{}
	Given the same relation
	\begin{align*}
		\text{Wants} \subseteq U \times U
	\end{align*}
	we define
	\begin{align*}
		\text{WantedBy}(b) := \{ a \in U \mid (a, b) \in \text{Wants} \}
	\end{align*}
	as the *reverse mapping* of the ``Wants'' relation. In unit files, specifying
	\texttt{WantedBy=target} means that \texttt{(target, unit)} is added to the
	``Wants'' relation.

	This declaration is used to construct the dependency graph at install time,
	via \texttt{systemctl enable}.

	Mostly used for targets.
\end{definition}

\section{Requires}
\begin{definition}{requires predicate}{}
	list of units this unit depends on which must be started as well.
\end{definition}

\section{Conflicts}
\begin{definition}{conflicts predicate}{}
	list of units this unit depends on which must be started as well.
\end{definition}

\section{After}
\begin{definition}{after predicate}{}
	Given a relation
	\begin{align*}
		\text{After} \subseteq U \times U
	\end{align*}
	we define the systemd behavior as:

	\begin{quote}
		If $(a, b) \in \text{After}$, then systemd ensures that if both $a$ and
		$b$ are scheduled to start, then $\mathsf{start}(b)$ completes
		(successfully or not) before $\mathsf{start}(a)$ begins.
	\end{quote}

	This relation governs start-time ordering but imposes no dependency: systemd
	does not automatically start $b$ when starting $a$.

	Thus, ``After'' is a partial order constraint on the timeline of unit
	startups, orthogonal to dependency relations like ``Wants''.
\end{definition}



\chapter{Units}
\section{Service}
Service units are the work-horse of systemd. Only within service units, we have
a type argument.

\begin{definition}{Templated Units}{}
	Templated units in \texttt{systemd} are unit files whose names contain an
	\texttt{@} symbol, such as \texttt{myservice@\{instance\}.service}. They define
	a generic structure that can be instantiated with specific identifiers. When an
	instance like \texttt{myservice@foo.service} is started, the placeholder is
	replaced with \texttt{foo}, and the unit can refer to this instance string using
	\texttt{\%i} (or \texttt{\%I} for the unescaped form). This enables modular,
	reusable configurations across services or users.
\end{definition}

\begin{intuition}{Templated Unit}{}
	A templated unit is essentially a $\lambda$-abstraction over a systemd unit
	file. Applying it with an instance name produces a concrete unit named
	\texttt{unit@\textit{instance}.service}, where \textit{instance} is the
	argument.
\end{intuition}


\begin{table}[h]
	\centering
	\begin{tabular}{@{}ll@{}}
		\toprule
		\textbf{Type}    & \textbf{Meaning}                                             \\ \midrule
		\texttt{simple}  & Default. Start service and assume it's ready immediately.    \\
		\texttt{exec}    & Like \texttt{simple}, but tracks the real PID of the binary. \\
		\texttt{forking} & Assumes service forks; parent exits when setup is done.      \\
		\texttt{oneshot} & For short-lived commands (e.g. setup); blocks until done.    \\
		\texttt{notify}  & Waits for \texttt{READY=1} via \texttt{sd\_notify()}.        \\
		\texttt{dbus}    & Waits until service acquires a specific D-Bus name.          \\
		\bottomrule
	\end{tabular}
	\caption{Valid \texttt{Type=} values for \texttt{.service} units}
\end{table}

Use notify when you want to control when the unit is considered started based on
your custom logic.

\section{Socket}

\section{Timer}

\section{Target}
\begin{definition}{Target}{}
	Targets are grouping units in systemd. They have no associated process;
	their only role is to aggregate other units for synchronization and
	dependency management.
\end{definition}

A target is considered active when all of its Requires= or Wants= are started
in the systemd sense---activating, active, or deactivating.

It doesn’t run code. It doesn't have a process. It just becomes active as soon
as the units it gathers are active.

\begin{table}[h]
	\centering
	\begin{tabular}{ll}
		\toprule
		\textbf{Target} & \textbf{Description}                              \\
		\midrule
		default         & Boot entry point; usually links to another target \\
		multi-user      & Non-graphical multi-user system (like runlevel 3) \\
		graphical       & Multi-user + graphical session (like runlevel 5)  \\
		rescue          & Single-user mode with essential services          \\
		emergency       & Minimal shell with almost nothing started         \\
		sysinit         & Early boot: mounts, udev, swap, tmpfiles          \\
		basic           & After sysinit; sockets, timers, dbus, etc.        \\
		network-online  & Network stack is fully configured                 \\
		shutdown        & Synchronization point during shutdown             \\
		\bottomrule
	\end{tabular}
	\caption{Key systemd targets and their purposes}
\end{table}

\begin{center}
	\begin{tabular}{lc}
		\toprule
		type      & resource control \\
		\midrule
		automount &                  \\
		device    &                  \\
		mount     &                  \\
		path      &                  \\
		scope     & \checkmark       \\
		service   & \checkmark       \\
		slice     & \checkmark       \\
		snapshot  &                  \\
		socket    &                  \\
		swap      &                  \\
		target    &                  \\
		timer     &                  \\
		\bottomrule
	\end{tabular}
\end{center}

\chapter{Targets}

\begin{definition}{target}{}
	\begin{alist}
		\item unit that lists dependencies on other targets
	\end{alist}
\end{definition}

\begin{tabular}{@{}ll@{}}
	\toprule
	\textbf{Target}                & \textbf{Purpose}                                                 \\
	\midrule
	\texttt{default.target}        & Main entry point after boot (usually symlinks to another target) \\
	\texttt{multi-user.target}     & Non-graphical system, networking, users, services                \\
	\texttt{graphical.target}      & Like multi-user + graphical session (e.g. display manager)       \\
	\texttt{network.target}        & Basic network stack is up (may be too early for apps)            \\
	\texttt{network-online.target} & Network is fully configured (for apps needing connectivity)      \\
	\texttt{sysinit.target}        & Early boot initialization (mounts, swap, etc.)                   \\
	\texttt{basic.target}          & After sysinit; before most services                              \\
	\texttt{rescue.target}         & Single-user mode with minimal services                           \\
	\texttt{emergency.target}      & Minimal shell, no services (critical recovery)                   \\
	\bottomrule
\end{tabular}


\chapter{Service Config}

\begin{definition}{Service Section}{}
	Let $u \in U$ be a systemd unit of type \texttt{.service}.

	The \texttt{[Service]} section defines the **execution behavior** of $u$:
	a finite map
	\begin{align*}
		\text{Service} : K \to V
	\end{align*}
	where $K$ is a set of predefined keys (e.g. \texttt{ExecStart}, \texttt{Restart}),
	and $V$ are values interpreted by systemd at runtime.

	This section governs the transition behavior of the unit’s state predicate $\mathsf{started}(u)$,
	including:
	\begin{itemize}
		\item when and how the unit is started
		\item what process it runs
		\item what happens on failure
	\end{itemize}
\end{definition}

\chapter{Sandbox}

\begin{tabular}{ll}
	\toprule
	\textbf{Option}                & \textbf{Description}                                                                            \\
	\midrule
	\texttt{CapabilityBoundingSet} & Limits kernel capabilities available to the service.                                            \\
	\texttt{DynamicUser}           & a transient user at runtime, Removed when the service stops.                                    \\
	\texttt{InaccessiblePaths}     & Makes specific paths inaccessible.                                                              \\
	\texttt{PrivateDevices}        & Restricts access to physical devices.                                                           \\
	\texttt{PrivateNetwork}        & Isolates in private network namespace.                                                          \\
	\texttt{PrivateTmp}            & Isolates \texttt{/tmp} and \texttt{/var/tmp} for the service.                                   \\
	\texttt{PrivateUsers}          & Hides the system’s user table from the service.                                                 \\
	\texttt{ProtectControlGroups}  & Prevents manipulating cgroups from inside the service.                                          \\
	\texttt{ProtectHome}           & Restricts access to user home directories.                                                      \\
	\texttt{ProtectKernelTunables} & Prevents modifying kernel tunables in \texttt{/proc/sys} and similar.                           \\
	\texttt{ProtectSystem}         & Makes parts of the file system read-only. Values: \texttt{yes}, \texttt{full}, \texttt{strict}. \\
	\texttt{ReadOnlyPaths}         & Makes specific paths read-only.                                                                 \\
	\texttt{RootDirectory}         & Changes the root directory for the service (like \texttt{chroot}).                              \\
	\texttt{User}                  & Runs service as specified (existing) user.                                                      \\
	\bottomrule
\end{tabular}


\chapter{Design Patterns}

\section{If This Then That}
Many systemd units follow an *if-then* structure, where the *then* is a
\texttt{.service} unit, and the *if* is an event unit. Common event types include
\texttt{.path} (file changes), \texttt{.mount} (mount state changes),
\texttt{.socket} (socket activation), \texttt{.timer} (scheduled time),
and \texttt{.device} (hardware changes). These trigger the corresponding
service via the \texttt{Unit=} directive.

\section{After plus Wants}
The ``wants'' relation

\chapter{Files}
\begin{table}[h]
	\centering
	\begin{tabular}{@{}lll@{}}
		\toprule
		\textbf{Description}          & \textbf{Path}                                            & \textbf{Scope} \\ \midrule
		Admin overrides (persistent)  & \texttt{/etc/systemd/system/}                            & System         \\
		Runtime overrides (temporary) & \texttt{/run/systemd/system/}                            & System         \\
		Package default units         & \texttt{/lib/systemd/system/} or \texttt{/nix/store/...} & System         \\
		User config overrides         & \texttt{\$XDG\_CONFIG\_HOME/systemd/user/}               & User           \\
		User runtime units            & \texttt{\$XDG\_RUNTIME\_DIR/systemd/user/}               & User           \\
		User package units            & \texttt{\$XDG\_DATA\_DIRS/systemd/user/}                 & User           \\
		\bottomrule
	\end{tabular}
	\caption{systemd unit file lookup hierarchy}
\end{table}


\chapter{Systemd Components}

\begin{tabular}{lll}
	\toprule
	\textbf{Component} & \textbf{Unit File}                      & \textbf{Type}     \\
	\midrule
	systemd (PID 1)    & N/A                                     & Special           \\
	systemd-journald   & \texttt{systemd-journald.service}       & \texttt{.service} \\
	systemd-logind     & \texttt{systemd-logind.service}         & \texttt{.service} \\
	systemd-udevd      & \texttt{systemd-udevd.service}          & \texttt{.service} \\
	systemd-networkd   & \texttt{systemd-networkd.service}       & \texttt{.service} \\
	systemd-resolved   & \texttt{systemd-resolved.service}       & \texttt{.service} \\
	systemd-timesyncd  & \texttt{systemd-timesyncd.service}      & \texttt{.service} \\
	systemd-homed      & \texttt{systemd-homed.service}          & \texttt{.service} \\
	systemd-tmpfiles   & \texttt{systemd-tmpfiles-setup.service} & \texttt{.service} \\
	systemctl (CLI)    & N/A                                     & Binary            \\
	\bottomrule
\end{tabular}

\chapter{Overrides}

% \printbibliography{}
% \printindex{}
\end{document}
