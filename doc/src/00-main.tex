\documentclass[openany, 12pt]{book}
\input{preamble}
\title{SystemD}
\author{Idris}
\date{June 2025}

%chktex-file 1

\begin{document}
\tableofcontents

\chapter{Background}

Let
\begin{align*}
	U = \set{u_1, u_2, \dots, u_n}
\end{align*}

be the finite set of all systemd units in the system. Each element $u_i \in U$
represents a unique systemd unit, such as a service, target, socket, or timer.

We define dependency and ordering relationships as binary relations on $U
	\times U$, which we will formalize in the following sections.

\begin{definition}{unit states}{}
	Let $S$ be the set of possible states of a unit:
	\begin{align*}
		S = \set{ \mathsf{inactive},\ \mathsf{activating},\ \mathsf{active},\ \mathsf{deactivating},\ \mathsf{failed}}
	\end{align*}
	Define a state function
	\begin{align*}
		a : U \to S
	\end{align*}
	which maps each unit to its current state.
\end{definition}

\begin{definition}{started predicate}{}
	Let
	\begin{align*}
		\mathsf{started}(u) \iff a(u) \in \set{ \mathsf{activating},\ \mathsf{active},\ \mathsf{deactivating}}
	\end{align*}
	This captures units that are in the process of starting, running, or
	stopping — i.e., **not inactive or failed**.
\end{definition}

\begin{definition}{wants predicate}{}
	Given a relation
	\begin{align*}
		\text{Wants} \subseteq U \times U
	\end{align*}
	we define the systemd behavior as:

	\begin{quote}
		If $\mathsf{started}(a)$ becomes true, then for each $(a, b) \in
			\text{Wants}$, systemd attempts $\mathsf{start}(b)$, regardless of
		whether $\mathsf{start}(b)$ eventually succeeds or fails.
	\end{quote}

	This defines `Wants` as a rule over the state predicate $\mathsf{started}$,
	rather than an abstract dependency graph.
\end{definition}

% \printbibliography{}
% \printindex{}
\end{document}
